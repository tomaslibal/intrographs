\begin{filecontents*}{rescommgraph.bib}
  @misc{weisstein2008antiprism,
    author="Eric W. {Weisstein}",
    title={Antiprism Graph},
    howpublished={http://mathworld.wolfram.com/AntiprismGraph.html},
    note={Accessed: 19 September 2015}
  }
  @book{balakrishnan1996,
  author="V. K. {Balakrishnan}",
  title={Introductory Discrete Mathematics},
  publisher={Dover Publications},
  year={1996}
  }
\end{filecontents*}

\documentclass[11pt]{article}

\usepackage{tikz}
\usepackage{caption}
\usepackage{subcaption}

\usepackage[backend=biber]{biblatex}
\addbibresource{resplanar.bib}

\title{Common Graphs and Graph Types}
\author{Tomas Libal}
\date{September 21, 2015}

\begin{document}

\maketitle

\section{Graphs, digraphs and multigraphs}

Let graph $ G = (V, E) $ have a vertex set $ V $ and an edge set $ E $. The vertex set is a non-empty set of distinct 
vertices $ v_1, v_2, ..., v_k $ where $ k = |V| $ and the edge set, which can be empty, contains subsets of two 
vertices $ \{v_s, v_t\} \in V $, and $ s, t\ \in \{1, 2, ..., k\} $.

\tikzstyle{vertex}=[circle,fill=yellow!50,minimum size=20pt,inner sep=0pt]
\tikzstyle{edge}=[draw,thick,-]

\begin{figure}[ht]
  \caption{Geometrical drawing of a graph}
  \label{fig:graph1}
\begin{tikzpicture}[scale=1.0, auto, swap]

  \node[vertex] (vertex1) at (5, 5) {v1};
  \node[vertex] (vertex2) at (2, 3) {v2};
  \node[vertex] (vertex3) at (2, 7) {v3};
  \node[vertex] (vertex4) at (7, 6) {v4};
  \node[vertex] (vertex5) at (3, 2) {v5};

\path[edge] (vertex1) edge (vertex2);
\path[edge] (vertex1) edge (vertex3);
\path[edge] (vertex3) edge (vertex4);
\path[edge] (vertex4) edge (vertex1);
\path[edge] (vertex4) edge (vertex3);
\path[edge] (vertex2) edge (vertex5);
\path[edge] (vertex2) edge (vertex3);
\end{tikzpicture}
\end{figure}

If there are no self-loops, that is, there are no edges incident on sets $ \{ v_s, v_s\} \in V, and s \in {1, 2, ..., k} $ 
then we call the graph \textit{simple}.

If some edges are incident on the same set of vertices then we call the object a \textit{multigraph} as can be seen in 
Figure~\ref{fig:multi}.

\begin{figure}[ht]
  \caption{Embedding of a multigraph on a plane}
  \label{fig:multi}
  \begin{tikzpicture}
    \node[vertex] (v1) at (3, 3) {m1};
    \node[vertex] (v2) at (1, 1) {m2};
    \node[vertex] (v3) at (5, 1) {m3};

    \path[edge] (v1) edge (v2);
    \path[edge] (v1) edge (v3);
    \path[edge] (v2) edge [bend right] (v1);
    \path[edge] (v2) edge [bend left] (v1);
  \end{tikzpicture}
\end{figure}

In \textit{directed graphs} or \textit{digraphs} the edge set $ E $ consists of ordered pairs $ \{(v_s, v_t) | s,t \in \{1, 2, ..., k\}\} $ 
so each edges has a direction leaving node $ v_j $ to $ v_k $.

\section{Some common graph types}

Antiprism graph "is a graph corresponding to the skeleton of an antiprism"~\autocite{weisstein2008antiprism} (see Figures~\ref{fig:3antiprism} and~\ref{fig:4antiprism}.

\tikzstyle{smallvertex}=[circle,fill=red!75,minimum size=5pt,inner sep=0pt]
\tikzstyle{lightedge}=[color=black!40]

\begin{figure}[ht]
\centering
\begin{subfigure}{0.3\textwidth}
  \caption{3-antiprism graph}
  \label{fig:3antiprism}
  \begin{tikzpicture}[scale=0.5]
      \node[smallvertex] (A) at (0, 6) {};
      \node[smallvertex] (B) at (0, 0) {};
      \node[smallvertex] (C1) at (2, 3) {};
      \node[smallvertex] (C2) at (3, 2.5) {};
      \node[smallvertex] (C3) at (3, 3.5) {};
      \node[smallvertex] (C) at (6, 3) {};

      \path[lightedge] (A) edge (B);
      \path[lightedge] (A) edge (C1);
      \path[lightedge] (A) edge (C3);

      \path[lightedge] (B) edge (C1);
      \path[lightedge] (B) edge (C2);
      \path[lightedge] (B) edge (C);

      \path[lightedge] (C1) edge (C2);
      \path[lightedge] (C2) edge (C3);
      \path[lightedge] (C1) edge (C3);

      \path[lightedge] (C) edge (A);
      \path[lightedge] (C) edge (C2);
      \path[lightedge] (C) edge (C3);

  \end{tikzpicture}
\end{subfigure}%
\hfill
\begin{subfigure}{0.3\textwidth}
	\caption{4-antiprism graph}
	\label{fig:4antiprism}
	\begin{tikzpicture}[scale=0.5]
      \node[smallvertex] (A) at (2.5, 6) {};
      \node[smallvertex] (B) at (0, 3) {};

      \node[smallvertex] (C1) at (2, 2.5) {};
      \node[smallvertex] (C2) at (3, 2.5) {};
      \node[smallvertex] (C3) at (3, 3.5) {};
      \node[smallvertex] (C4) at (2, 3.5) {};

      \node[smallvertex] (C) at (2.5, 0) {};
      \node[smallvertex] (D) at (5, 3) {};

      \path[lightedge] (A) edge (C4);
      \path[lightedge] (A) edge (C3);
      \path[lightedge] (A) edge (D);
      \path[lightedge] (A) edge (B);

      \path[lightedge] (B) edge (C1);
      \path[lightedge] (B) edge (C4);
      \path[lightedge] (B) edge (C);

      \path[lightedge] (C1) edge (C2);
      \path[lightedge] (C2) edge (C3);
      \path[lightedge] (C1) edge (C4);
      \path[lightedge] (C4) edge (C3);

      \path[lightedge] (C) edge (D);
      \path[lightedge] (C) edge (C1);
      \path[lightedge] (C) edge (C2);

      \path[lightedge] (D) edge (C2);
      \path[lightedge] (D) edge (C3);
	\end{tikzpicture}
\end{subfigure}
\caption{Antiprism graphs}
\end{figure}

Bipartite graph's vertex set can be partitioned into two disjoint sets~\autocite[122]{balakrishnan1996} in which vertices from one set can be adjacent only to vertices in the other set. This is a special case of the \textit{k}--partite graph where $ k = 2 $ (see Figure~\ref{fig:bipartitegroup}.

\tikzstyle{smvertex2}=[circle,fill=blue!75,minimum size=5pt, inner sep=0pt]
\begin{figure}[ht]

	\begin{subfigure}{0.3\textwidth}
      \caption{A bipartite graph with 7 nodes}
      \label{fig:bipartite1}

      \begin{tikzpicture}
          \node[smallvertex] (A) at (0, 6) {};
          \node[smallvertex] (B) at (0, 4) {};
          \node[smallvertex] (C) at (0, 2) {};
          \node[smallvertex] (D) at (0, 0) {};

          \node[smvertex2] (X) at (3, 5) {};
          \node[smvertex2] (Y) at (3, 3) {};
          \node[smvertex2] (Z) at (3, 1) {};

          \path[lightedge] (A) edge (X);
          \path[lightedge] (A) edge (Z);
          \path[lightedge] (B) edge (X);
          \path[lightedge] (C) edge (Z);
          \path[lightedge] (C) edge (X);
          \path[lightedge] (D) edge (X);
          \path[lightedge] (D) edge (Y);
          \path[lightedge] (D) edge (Z);

      \end{tikzpicture}
    \end{subfigure}%
    \hfill
    \begin{subfigure}{0.3\textwidth}
    	\caption{A complete bipartite graph with 6 nodes}
     	\label{fig:bipartite1}

      \begin{tikzpicture}
          \node[smallvertex] (A) at (0, 6) {};
          \node[smallvertex] (B) at (0, 4) {};
          \node[smallvertex] (C) at (0, 2) {};

          \node[smvertex2] (X) at (3, 6) {};
          \node[smvertex2] (Y) at (3, 4) {};
          \node[smvertex2] (Z) at (3, 2) {};

          \path[lightedge] (A) edge (X);
          \path[lightedge] (A) edge (Y);
          \path[lightedge] (A) edge (Z);
          \path[lightedge] (B) edge (X);
          \path[lightedge] (B) edge (Y);
          \path[lightedge] (B) edge (Z);
          \path[lightedge] (C) edge (X);
          \path[lightedge] (C) edge (Y);
          \path[lightedge] (C) edge (Z);

      \end{tikzpicture}
    \end{subfigure}
    \caption{Bipartite graphs}
    \label{fig:bipartitegroup}
\end{figure}


\end{document}
