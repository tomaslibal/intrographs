\begin{filecontents*}{rescommgraph.bib}
  @misc{weisstein2008antiprism,
    author="Eric W. {Weisstein}",
    title={Antiprism Graph},
    howpublished={http://mathworld.wolfram.com/AntiprismGraph.html},
    note={Accessed: 19 September 2015}
  }
\end{filecontents*}

\documentclass[11pt]{article}

\usepackage{tikz}

\usepackage[backend=biber]{biblatex}
\addbibresource{resplanar.bib}

\title{Common Graphs and Graph Types}
\author{Tomas Libal}
\date{September 21, 2015}

\begin{document}

\maketitle

Let graph $ G = (V, E) $ have a vertex set $ V $ and an edge set $ E $. The vertex set is a non-empty set of distinct 
vertices $ n_1, n_2, ..., n_k $ where $ k = |V| $ and the edge set, which can be empty, contains subsets of two 
vertices $ \{n_j, n_k\} \in V $, and $ j, k\ \in \{1, 2, ..., k\} $.

\tikzstyle{vertex}=[circle,fill=yellow!50,minimum size=20pt,inner sep=0pt]
\tikzstyle{edge}=[draw,thick,-]

\begin{figure}[h]
  \caption{Geometrical drawing of a graph}
  \label{fig:graph1}
\begin{tikzpicture}[scale=1.0, auto, swap]

  \node[vertex] (vertex1) at (5, 5) {v1};
  \node[vertex] (vertex2) at (2, 3) {v2};
  \node[vertex] (vertex3) at (2, 7) {v3};
  \node[vertex] (vertex4) at (7, 6) {v4};
  \node[vertex] (vertex5) at (3, 2) {v5};

\path[edge] (vertex1) edge (vertex2);
\path[edge] (vertex1) edge (vertex3);
\path[edge] (vertex3) edge (vertex4);
\path[edge] (vertex4) edge (vertex1);
\path[edge] (vertex4) edge (vertex3);
\path[edge] (vertex2) edge (vertex5);
\path[edge] (vertex2) edge (vertex3);
\end{tikzpicture}
\end{figure}

If there are no self-loops, that is, there are no edges incident on sets $ \{ n_j, n_j\} \in V, j \in {1, 2, ..., k} $ 
then we call the graph \textit{simple}.

If some edges are incident on the same set of vertices then we call the object a \textit{multigraph} as can be seen in 
Figure~\ref{fig:multi}.

\begin{figure}[h]
  \caption{Embedding of a multigraph on a plane}
  \label{fig:multi}
  \begin{tikzpicture}
    \node[vertex] (v1) at (3, 3) {m1};
    \node[vertex] (v2) at (1, 1) {m2};
    \node[vertex] (v3) at (5, 1) {m3};

    \path[edge] (v1) edge (v2);
    \path[edge] (v1) edge (v3);
    \path[edge] (v2) edge [bend right] (v1);
    \path[edge] (v2) edge [bend left] (v1);
  \end{tikzpicture}
\end{figure}

In \textit{directed graphs} or \textit{digraphs} the edge set $ E $ consists of ordered pairs $ \{(n_j, n_k) | j,k \in \{1, 2, ..., k\}\} $ 
so each edges has a direction leaving node $ n_j $ to $ n_k $.

\end{document}
